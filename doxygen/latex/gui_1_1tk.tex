\hypertarget{gui_1_1tk_i}{}\section{Introduction}\label{gui_1_1tk_i}
gui\-::tk is a simple one-\/file C++ \hyperlink{namespaceGUI}{G\-U\-I} toolkit for use with arbitrary memory framebuffers.\hypertarget{gui_1_1tk_f}{}\section{Features}\label{gui_1_1tk_f}
\begin{DoxyItemize}
\item small source and binary code size \item reasonable performance and memory usage \item comfortable usage \item suitable for embedded usage\-: integer math only \item extensibility via O\-O \item non-\/intrusive\-: can be integrated with any event mechanism of your liking \item no dependencies apart from standards-\/conformant A\-N\-S\-I C++ (including a little S\-T\-L) \item support for different encodings, single-\/ and multibyte \item flexible font support\end{DoxyItemize}
\hypertarget{gui_1_1tk_o}{}\section{Overview}\label{gui_1_1tk_o}
The toolkit draws on a surface you provide, using any size or pixel format. Create a \hyperlink{classGUI_1_1Screen}{G\-U\-I\-::\-Screen} with the buffer to draw on, then pass that object (or a \hyperlink{classGUI_1_1Window}{G\-U\-I\-::\-Window} created from it) to all other widgets' constructors.

It doesn't provide an own event loop. Instead, it relies on you passing events and updating the screen regularly. This way, it can easily be integrated with any event loop available (S\-D\-L, Qt, glib, X11, Win32, selfmade, ...)

Many functions and concepts were taken from other well-\/known toolkits, so if you know Qt, Java or wx\-Windows, you may encounter (intentional) similarities. However, simplicity of code has been given priority over maximum optimization and pixel-\/exact replication of their appearance.

A note about font support and encodings\-: All text-\/related functions use templates to support any encoding. Fonts define the charset.\hypertarget{gui_1_1tk_g}{}\section{Getting Started}\label{gui_1_1tk_g}
gui\-::tk contains an S\-D\-L adapter, so if your program is already using S\-D\-L, things are really easy. The rough sequence to get up and running is\-:


\begin{DoxyCode}
 // setup a suitable video mode with SDL
 SDL_surface *screen = SDL_SetVideoMode(640, 480, 32, SDL_SWSURFACE);

 // add a default font, you will most probably need it
 GUI::Font::addFont("default",new GUI::BitmapFont(testfont,14,10));

 // create the root-level screen object, the parent of all top-level windows
 GUI::ScreenSDL guiscreen(screen);

 // create any amount of toplevel windows you like
 GUI::ToplevelWindow *frame = new GUI::ToplevelWindow(&guiscreen, 205, 100, 380
      , 250, "gui::tk example");

 // add some GUI elements to the toplevel window
 GUI::Button *b = new GUI::Button(frame,8,20,"Quit");

 // Define and add an event handler
 struct quit : public GUI::ActionEventSource_Callback {
     void actionExecuted(GUI::ActionEventSource *b, const GUI::String &arg) {
         exit(0);
     }
 } ex;
 b->addActionHandler(&ex);

 // Enter an event loop, calling update() regularly.
 SDL_Event event;
 while (1) {
     while (SDL_PollEvent(&event)) {
         if (!guiscreen.event(event)) { // gui::tk didn't handle that event
             if (event.type == SDL_QUIT) exit(0);
         }
     }

     guiscreen.update(4); // 4 ticks = 40ms
     SDL_UpdateRect(screen, 0, 0, screen->w, screen->h);

     SDL_Delay(40); // wait 40ms
 }
\end{DoxyCode}


Look at \hyperlink{gui__tk_8h}{gui\-\_\-tk.\-h} for more detailed documentation and reference of all classes.

A test program that shows off the capabilities of gui\-::tk is available as part of \hyperlink{gui__tk_8cpp}{gui\-\_\-tk.\-cpp}.\hypertarget{gui_1_1tk_l}{}\section{License}\label{gui_1_1tk_l}
Copyright (C) 2005-\/2007 Jörg Walter

gui\-::tk is free software; you can redistribute it and/or modify it under the terms of the G\-N\-U General Public License as published by the Free Software Foundation; either version 3 of the License, or (at your option) any later version.

gui\-::tk is distributed in the hope that it will be useful, but W\-I\-T\-H\-O\-U\-T A\-N\-Y W\-A\-R\-R\-A\-N\-T\-Y; without even the implied warranty of M\-E\-R\-C\-H\-A\-N\-T\-A\-B\-I\-L\-I\-T\-Y or F\-I\-T\-N\-E\-S\-S F\-O\-R A P\-A\-R\-T\-I\-C\-U\-L\-A\-R P\-U\-R\-P\-O\-S\-E. See the G\-N\-U General Public License for more details.

You should have received a copy of the G\-N\-U General Public License along with this program. If not, see \href{http://www.gnu.org/licenses/}{\tt http\-://www.\-gnu.\-org/licenses/} 